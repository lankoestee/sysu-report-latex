\documentclass[12pt]{ctexart}
\usepackage{geometry,tikz,graphicx,tabularx,float,xcolor,color,colortbl,cite,booktabs,colortbl}
\usepackage{subcaption,multirow,fancyhdr,gensymb,amsmath,pythonhighlight,multicol,makecell}
\usepackage[colorlinks,bookmarksopen,bookmarksnumbered]{hyperref}
% minted包用于展示代码,需要安装python和pygments
\usepackage[cache=false]{minted}
\usepackage[T1]{fontenc}

% ---------- 个人信息改这里 ----------
\newcommand{\school}{一个学院}
\newcommand{\major}{一个年级专业}
\newcommand{\righthead}{一个页眉标题}
\newcommand{\maintitle}{一个大标题}
\newcommand{\course}{一个课程}
\newcommand{\teacher}{一个老师}
\newcommand{\expnumber}{一个报告类型}
\newcommand{\name}{一个名字}
\newcommand{\id}{一个学号}
% ---------- 个人信息改这里 ----------

% \usemintedstyle{paraiso}
\setlength{\parindent}{2em}
% 这里的页边距是word文档的默认页边距
\geometry{left=2.54cm,right=2.54cm,top=2.54cm,bottom=2.54cm}
\geometry{headsep=1.2cm}
% \graphicspath{ {./figure/} }
\pagestyle{fancy}
\lhead{\includegraphics[width=34.7mm]{figure/badge-horizonal.pdf}}
\rhead{\righthead}
\makeatletter
\newcommand\dlmu[2][4cm]{\hskip1pt\underline{\hb@xt@ #1{\hss#2\hss}}\hskip3pt}
\makeatother
\hypersetup{colorlinks=true,linkcolor=black,citecolor=green}
% 这里设置了一个浅浅的灰色作为代码背景色
\definecolor{codebg}{rgb}{0.95,0.95,0.95}
\setminted{
    bgcolor=codebg,
    linenos,
    breaklines,
    breakanywhere,
}

\begin{document}
\begin{titlepage}
    \centering
    % \quad\\[0.2cm]
    \includegraphics[width=9cm]{figure/badge.pdf}
    \\[1cm]\textbf{\huge{\maintitle}}\\[1cm]
    \begin{minipage}{11.1cm}
        \centering
        \begin{flushleft} \Large
            \makebox[3cm][s]{\textbf{\fangsong{学院:}}}\dlmu[8cm]{\school}\\[0.3cm]
            \makebox[3cm][s]{\textbf{\fangsong{年级专业:}}}\dlmu[8cm]{\major}\\[0.3cm]
            \makebox[3cm][s]{\textbf{\fangsong{课程:}}}\dlmu[8cm]{\course}\\[0.3cm]
            \makebox[3cm][s]{\textbf{\fangsong{指导老师:}}}\dlmu[8cm]{\teacher}\\[0.3cm]
            \makebox[3cm][s]{\textbf{\fangsong{报告编号:}}}\dlmu[8cm]{\expnumber}\\[0.3cm]
            \makebox[3cm][s]{\textbf{\fangsong{小组成员:}}}\dlmu[8cm]{\name}\\[0.3cm]
            \makebox[3cm][s]{\textbf{\fangsong{学号:}}}\dlmu[8cm]{\id}\\[0.3cm]
            \makebox[3cm][s]{\textbf{\fangsong{日期:}}}\dlmu[8cm]{\today}\\[0.3cm]
        \end{flushleft}
    \end{minipage}
    \newpage
    \tableofcontents
    \newpage
\end{titlepage}
\hypersetup{linkcolor=red}

% ---------- 正文开始 ----------

% 可以直接在这里写,也可以在其他tex文件中写然后使用\include或者\input命令进行引入

% 下面是一些用作展示的代码,用的时候删掉就好了
\section{一级标题}
\subsection{二级标题}
\subsubsection{三级标题}
\begin{minted}{python}
def func():
    print("Hello, world!")
\end{minted}
下面展示了一个图片还有一个表格,如图\ref{fig:badge}所示,表\ref{tab:table1}是一个简单的三线表。
\begin{figure}[ht]
    \centering
    \includegraphics[width=0.5\textwidth]{figure/badge-horizonal.pdf}
    \caption{这是一个图片}
    \label{fig:badge}
\end{figure}
\begin{table}[ht]
    \centering
    \caption{这是简单的三线表}
    \label{tab:table1}
    \begin{tabular}{ccc}
        \toprule
        \textbf{A} & \textbf{B} & \textbf{C} \\
        \midrule
        1 & 2 & 3 \\
        4 & 5 & 6 \\
        7 & 8 & 9 \\
        \bottomrule
    \end{tabular}
\end{table}
这是一个简单的行内公式$E=mc^2$,这是一个简单的行间公式:
\begin{equation}
    E=mc^2
\end{equation}
行内的代码可以这样写\mintinline{python}|print('Hello World')|
\section{再一个一级标题}
\subsection{再一个二级标题}

% ---------- 正文结束 ----------

% 如果有参考文献的话,可以取消下面几行的注释
% \newpage
% \addcontentsline{toc}{section}{参考文献}
% \bibliographystyle{plain}
% % 这里的reference就是你的bib文件的名字
% \bibliography{reference}
\end{document}